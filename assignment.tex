%%% Template originaly created by Karol Kozioł (mail@karol-koziol.net) and modified for ShareLaTeX use

\documentclass[a4paper,12pt]{article}

\usepackage[T1]{fontenc}
\usepackage[utf8]{inputenc}
\usepackage{graphicx}
\usepackage{xcolor}
\usepackage{indentfirst}
\renewcommand\familydefault{\sfdefault}
\usepackage{tgheros}


\usepackage{amsmath,amssymb,amsthm,textcomp}
\usepackage{enumerate}
\usepackage{multicol}
\usepackage{tikz}

\usepackage{geometry}
\geometry{left=25mm,right=25mm,%
bindingoffset=0mm, top=20mm,bottom=20mm}


\linespread{1.5}

\newcommand{\linia}{\rule{\linewidth}{0.5pt}}


% my own titles
\makeatletter
\renewcommand{\maketitle}{
\begin{center}
\vspace{2ex}
{\huge \textsc{\@title}}
\vspace{1ex}
\\
\linia\\
\@author \hfill \@date
\vspace{4ex}
\end{center}
}
\makeatother
%%%

% custom footers and headers
\usepackage{fancyhdr}
\pagestyle{fancy}
\lhead{}
\chead{}
\rhead{}
\lfoot{Tarefa \textnumero{5} }
\cfoot{}
\rfoot{Página \thepage}
\renewcommand{\headrulewidth}{0pt}
\renewcommand{\footrulewidth}{0pt}
%



\begin{document}

\title{Tarefa 5 da disciplina de fundamentos de sistemas de informação.}

\author{Guilherme Elui de Souza - 11796152, Universidade de São Paulo}

\date{06/05/2020}

\maketitle

\section*{Resumo de \textit{As Empresas São Grandes Coleções de Processos} de José Ernesto Lima Gonçalves.}

O artigo começa por discutir uma definição tradicional e estrita do que são processos, conceito que determina entradas e saídas claras. A seguir, tal fluxo de trabalho é caracterizado como apenas uma das formas de processo empresarial, dada a interdisciplinaridade característica da Administração de Empresas, e já que não abrange processos que não têm início e fim padrões. Sendo assim, amplia-se a noção de processo de trabalho para a maneira particular de realizar um determinado conjunto de tarefas, e portanto estão incluídos também o número de operadores, a distribuição do trabalho, a tecnlogia empregada, indicadores de eficácia e resultados aguardados. Um processo agora envolve endpoints, transformações, feedback e repetibilidade. O uso de tais atributos na definição do processo permite condições mais adequadas para a sua análise e gestão.
\\     \indent Logo após, os autores passam a apresentar três categorias de processos empresariais: os processos de negócio caracterizam a atuação da empresa e resultam em serviço voltado para um cliente externo, processos organizacionais possibilitam coordenação de subsistemas em busca de seu desempenho geral, o que gera suporte adequado aos processos de negócio, e processos gerenciais, que abrangem as ações de medição e ajuste do desempenho da organização.
\\     \indent Posteriormente, são expostas características fundamentais dos processos empresariais, os quais podem ser internos ou externos, inter ou intra-organizacionais, horizontais ou verticais. Outra qualidade importante dos processos é a denominada interfuncionalidade, que é dita do processo que atravessa as fronteiras das áreas funcionais. Além disso, processos de negócio são caracterizados por possuirem clientes, e dessa forma utilizam os recursos da organização para oferecer resultados objetivos para os mesmos. Em seguida, são listadas razões pelas quais os processos são importantes, dentre elas pode-se citar que eles são relevantes para definir o modelo de organização das pessoas e recursos da empresa, eles são a fonte das competências específicas da empresa e influenciam estratégias, produtos, estruturas e a indústria. Uma das mais importantes aplicações da ideia de processos é a simulação do funcionamento de novas formas operacionais de obtenção dos resultados da empresa. Por fim, outra aplicação importante é na implementação das mudanças previstas para a operacionalização de um novo processo.
\\     \indent A estrutura organizacional por processos surge como forma dominante para o século, substituindo o raciocínio compartimentado do paradigma funcional. O primeiro estágio é adotar a redistribuição dos recursos humanos e técnicos ao longo dos processos de negócios. O segundo estágio é dado com o surgimento de parcerias e redes de empresas.
\\     \indent 
A tecnologia possui papel fundamental no estudo dos processos. Ela influencia tanto a forma de realizar o trabalho como a maneira de gerenciá-lo. Muitas vezes, o processo obedece a uma seqüência estrita de atividades, ditada pela sua tecnologia característica ou pela própria lógica do trabalho.









\end{document}
