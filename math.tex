\documentclass[12pt,a4paper]{article}
\usepackage[utf8]{inputenc}
\usepackage[portuguese]{babel}
\usepackage[T1]{fontenc}
\usepackage{amsmath}
\usepackage{listings}
\usepackage{amsfonts}
\usepackage{amssymb}
\usepackage{amsthm}
\usepackage{xcolor}
\usepackage{textcomp}
\usepackage{enumitem}
\usepackage[mathscr]{euscript}

\begin{document}
\author{Guilherme Elui de Souza, \textnumero{00000000} }
\title{ACH2013 - Matemática Discreta \\
\large Prof. Name \\
\large Atividade X}
\maketitle

\section*{Seção 4.1.}
\subsection*{Exercício 10}
\begin{enumerate}[label=\textbf{\alph*.}]
\item 2 
\item 2
\item 1
\item 3
\item 3
\end{enumerate}
\subsection*{Exercício 16}
\begin{enumerate}[label=\textbf{\alph*.}]
\item V
\item F, \quad a relação $\in$ deve ser entre um elemento e um conjunto.
\item F, \quad a relação $\in$ deve ser entre um elemento e um conjunto.
\item V
\item F, \quad a relação $\in$ deve ser entre um elemento e um conjunto.
\item F, \quad $B$ não possui elementos. $C$ possui um elemento, logo $B \neq C$.
\item V
\item F, \quad a relação $\in$ deve ser entre um elemento e um conjunto.
\item F, \quad $\{{\varnothing}\} \in D$ e $\{{\varnothing}\} \not\in A$
\end{enumerate}
\subsection*{Exercício 20}
$\cdot$ Resolvendo-se as equações:
\begin{center}
$\cos(\frac{x}{2})=0=\cos(\frac{\pi}{2})=cos(\frac{3\pi}{2})=\cos(\frac{5\pi}{2})= ... =\cos(\frac{\pi}{2} + k\pi)$, $k\in \mathbb{N}$ \\
\vspace{0.5cm}
$\sin(x)=0=\sin(\pi)=sin(2\pi)=\sin(3\pi)= ... =\sin(k\pi)$, $k\in \mathbb{N}$ \\
\end{center}
$\cdot$ O conjunto solução $A$ da primeira equação é:
\begin{center}
$A= \{\pi,3\pi,5\pi,...\}$
\end{center}
$\cdot$ O conjunto solução $B$ da segunda equação é:
\begin{center}
$B = \{0,\pi,2\pi,3\pi,...\}$
\end{center}
$\cdot$ Logo é possível observar que $A \subseteq B$
\subsection*{Exercício 22}
\begin{enumerate}[label=\textbf{\alph*.}]
\item F, \quad $(\varnothing \subset A)$
\item F, \quad $(\varnothing \neq \{\varnothing\}) $
\item V
\item F, \quad $A$ e $C$ podem ser iguais e ambos diferentes de $B$
\item F, \quad $A=\{0\}, B=\{\{0\},1\}, C=\{\{0\},3\}$
\end{enumerate}
\subsection*{Exercício 26}
$\cdot$ Prova por indução \\ \\
$\cdot$ \textit{Caso base:} $n=3$. \\ Um conjunto $A = \{a,b,c\}$ com três elementos tem exatamente \textbf{um} subconjunto com esses mesmos três elementos.
\begin{center}
$n(n-1)(n-2)/6 \to 3(3-1)(3-2)/6 = 1$
\end{center}
$\cdot$ \textit{Hipótese indutiva:} \\ Assumir que um conjunto com $k$ elementos possui $k(k-1)(k-2)/6$ subconjuntos com exatamente três elementos.\\ \\
$\cdot$ \textit{Passo indutivo:} queremos provar a propriedade verdadeira para $k+1$. \\
Seja portanto $x$ um membro de um conjunto com $k+1$ elementos. Se retirarmos $x$ desse conjunto teremos um conjunto com $k$ elementos, e que por hipótese indutiva possui $k(k-1)(k-2)/6$ subconjuntos com três elementos. O número de subconjuntos de três elementos que incluem $x$ pode ser obtido atráves do cálculo do número de subconjuntos com dois elementos (bastaria em seguida incluir $x$ aos subconjuntos). Conforme o exercício anterior, esse valor é dado por $k(k-1)/2$. Sendo assim, o número total de subconjuntos com três elementos pode ser obtido somando subconjuntos que não incluem x, com aqueles que incluem $x$. Logo,
\begin{center}
$\frac{k(k-1)(k-2)}{6} + \frac{k(k-1)}{2} = \frac{k(k-1)(k-2)}{6} + \frac{3k(k-1)}{6} = $
\end{center}
\begin{center}
$= \frac{k(k-1)[(k-2)+(3)]}{6} = \frac{k(k-1)(k+1)}{6}$
\end{center}
\qed

\subsection*{Exercício 35}
\begin{equation*}
\begin{split}
x\in A & \iff \{x\}\in \wp (A) \\
\wp (A) = \wp (B) & \implies \{x\} \in \wp (B) \\
& \iff x \in B \\
& \implies A \subseteq B
\end{split}
\end{equation*} \\
\begin{equation*}
\begin{split}
x\in B & \iff \{x\}\in \wp (B) \\
\wp (A) = \wp (B) & \implies \{x\} \in \wp (A) \\
& \iff x \in A \\
& \implies B \subseteq A
\end{split}
\end{equation*}
Sendo assim,
\begin{center}
$A \subseteq B \,\land\, B \subseteq A \implies A = B$
\end{center}

\subsection*{Exercício 39}
\begin{enumerate}[label=\textbf{\alph*.}]
\item Operação binária.
\item Operação binária.
\item Operação binária.
\item Não é uma operação, $\ln x$ não é definida para $x\leq 0$.
\end{enumerate}
\subsection*{Exercício 41}
\begin{enumerate}[label=\textbf{\alph*.}]
\item Não é uma operação, $x=0$ não é definida.
\item Operação binária.
\item Operação unária.
\item Operação binária.
\end{enumerate}
\subsection*{Exercício 55}
$A=$ números inteiros múltiplos de $3$ e maiores que $4$\\
$B=$ números inteiros pares \\
$C=$ números inteiros no intervalo fechado entre $10$ e $-10$
\begin{enumerate}[label=\textbf{\alph*.}]
\item $B'$.
\item $B \cap C $
\item $A\cap B $
\item $B'\cap C $
\item $B'\cap C' $
\end{enumerate}
\subsection*{Exercício 73}
$\cdot$ Prova por contradição \\ 

$\cdot$ Assuma por absurdo que se $A \cup B = A-B$, então $B \neq \varnothing$. \\
\begin{equation*}
\begin{split}
B \neq \varnothing & \iff \exists x  \in B \\
&\implies x \in A \cup B \\
x \in B & \implies  x\not\in A-B
\end{split}
\end{equation*}

$\cdot$ O que contradiz a equação inicial, e portanto conclui-se que a hipótese inicial estava errada.

\subsection*{Exercício 77}
$\cdot$ Prova por contradição que $(A \subseteq B) \iff (A\cap B' = \varnothing)$. \\ 

$\cdot$ $(\implies)$ Vamos assumir por absurdo $A\cap B' \neq \varnothing$
\begin{equation*}
\begin{split}
A\cap B' \neq \varnothing &\iff (\exists x)(x\in A)(x\not\in B) \\
&\iff (\exists x)(x\in A)(x\in B') \\
&\iff (\exists x)(x\in A\cap B') \Rightarrow\Leftarrow
\end{split}
\end{equation*} \\

$\cdot$ $(\impliedby)$ Vamos assumir por absurdo $A \not\subseteq B $
\begin{equation*}
\begin{split}
A\cap B' \neq \varnothing &\iff (\exists x)(x\in A)(x\not\in B) \\
&\iff (\exists x)(x\in A)(x\in B') \\
&\iff (\exists x)(x\in A\cap B') \Rightarrow\Leftarrow
\end{split} 
\end{equation*}

\section*{Seção 4.2.}
\subsection*{Exercício 8}
$\cdot$ Uma rota possível é $A \to B \to D$ e há $3.2=6$ modos de percorrer este caminho. A outra é $A \to C \to D$ e existem $2.4=8$ formas de percorrer este caminho. Como tratam-se de eventos disjuntos, pelo príncipio aditivo temos no total $8+6=14$ rotas possíveis.
\subsection*{Exercício 11}
$\cdot$ Sendo que o alfabeto possui $26$ letras e em um palíndromo com 5 letras teremos a repetição de duas delas, concluimos que existem $26^3 = 17576$ palíndromos com $5$ letras.
\subsection*{Exercício 19}
$\cdot$ Pelos princípio multiplicativo, $10.7.3.2.2.2 = 1680$ carros diferentes.
\subsection*{Exercício 34}
Como a string começa e termina com $0$, restam 6 espaços para decidir entre $0$ ou $1$. Portanto, temos $2^6 = 64$ strings possíveis.
\subsection*{Exercício 39}
No conjunto $A$(começam com $10$), temos $2^6$ possibilidades. No conjunto $B$($0$ na terceira posição), temos $2^7$ possibilidades. Já para o conjunto, $(A\cap B)$, temos $2^5$ possibilidades. Sendo assim, $|A\cup B| =|A|+|B|-|A\cap B| = 2^6 + 2^7 - 2^5 = 2^5(2+2^2-1)=160$.

\section*{Seção 4.3.}
\subsection*{Exercício 6}
$A =$ gostam de rock \\
$B =$ gostam de country \\
$C =$ gostam de música clássica \\ \\
$|A \cup B \cup C| = |A|+|B|+|C| - |A\cap B| - |A\cap C| -|B\cap C| + |A\cap B\cap C|$ \\
$24 = 14 + |B| + 17 - 11 - 9 - 13  + 8 \implies |B| = 18$

\subsection*{Exercício 10}
$A=$ clientes com checking accounts \\
$B=$ clientes com regular savings accounts \\
$C=$ clientes com market savings accounts
\subsubsection*{a.}
$|A \cup B \cup C| = |A|+|B|+|C| - |A\cap B| - |A\cap C| -|B\cap C| + |A\cap B\cap C|$ \\
$214 = 189 + 73 + 114 - 69 - 0 - |A\cap C|  + 0 \implies |A\cap C| = 93$
\subsubsection*{b.}
$|A-(B\cup C)| = |A| - |A\cap(B\cup C)| = |A| - |(A\cap B) \cup (A\cap C)|$ \\
$= |A| - (|A\cap B| + |A\cap C||) = 189 - (69+93) = 27 $
\subsection*{Exercício 14}
$A =$ múltiplos de $3$, \quad $|A| =\lfloor \frac{1000}{3}\rfloor = 333 $ \\ \\
$B =$ múltiplos de $7$, \quad $|B| =\lfloor \frac{1000}{7}\rfloor = 142 $ \\ \\
$|A\cap B| = \lfloor \frac{1000}{3\,.\,7}\rfloor = 47$ \\ \\
$|A\cup B| = |A| +|B| - |A\cap B| = 333+142-47=428  $
\subsection*{Exercício 26}
$\cdot$ Temos cinco somas (gavetas) possíveis: $2+20$, $4+18$, $6+16$, $8+14$ e $10+12$. O pior dos casos para obter a soma $22$, seria escolhermos cinco números e nenhum complementar que dê $22$. Por exemplo, no caso de escolhermos 2, 4, 6, 8 e 10. Entretanto, na \textbf{sexta} escolha com certeza seria formado algum dos pares que quando adicionados dão $22$. \textbf{Resposta}: $6$ números.


\end{document}
